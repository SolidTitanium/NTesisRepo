\chapter{An\'alisis Topol\'ogico de datos para Ciencia de Datos con la Libreria GUDHI}

En esta secci\'on, ilustraremos m\'etodos del ATD usando la libreria de Python
GUDHI\footnote{https://gudhi.inria.fr/python/latest} (Maria et al., 2014\cite{Maria2014})
junto con otras librerias populares como Numpy (Walt et al., 2011\cite{Walt2011}),
scikit-learn (Pedregosa et al., 2011)\cite{Pedregosa2011},
y pandas (McKinney, 2010\cite{McKinney2010}).
Esta secci\'on se enfoca en demostrar que las firmas topol\'ogicas del ATD
pueden ser facilmente calculadas y explotadas usando GUDHI.
Se pueden encontrar m\'as ejemplos en el tutorial de GUDHI en
GitHub\footnote{https://github.com/GUDHI/TDA-tutorial}.\medskip\medskip

\section{Bootstrap y Comparasi\'on de Configuraciones de Uni\'on de Prote\'inas}

Este ejemplo lo tomamos prestado de Kovacev-Nikolic et al (2016)\cite{Kovacev2016}.
En este art\'iculo, la homolog\'ia persistente es usada para analizar la uni\'on de
prote\'inas, y m\'as precisamente, compara las formas abiertas y cerradas de la
proteina de uni\'on a la maltosa (MBP), una biomol\'ecula que consiste de
$370$ residuos de amino\'acidos.