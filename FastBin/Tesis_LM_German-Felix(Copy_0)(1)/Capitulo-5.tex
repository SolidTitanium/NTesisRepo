\chapter{Aspectos Estad\'isticos de la Homolog\'ia Persistente}\label{chap:Cap5}

Por s\'i sola, la homolog\'ia persistente no toma en cuenta la naturaleza estoc\'astica de los datos
y la variabilidad intrinseca de las cantidades topol\'ogicas que infieren.
Buscamos ahora un acercamiento esta\'distico a la homolog\'ia persistente,
considerando que los datos son generados de alguna distribuci\'on desconocida.
Comenzamos dando varos resultados de consistencia para la inferencia de la homolog\'ia persistente.

\section{Resultados de Consistencia para la Homolog\'ia Persistente}

Sup\'ongase que observamos $n$ puntos $\cpar{X_{1},\dots,X_{n}}$ en un espacio m\'etrico
$\cpar{M,\rho}$ obtenidas i. i. d. de una medida de probabilidad desconocida $\mu$ con
soporte compacto $\mathbb{X}_{\mu}$ la distancia de Gromov-Hausdorff nos permite comparar
$\mathbb{X}_{\mu}$ con otros espacios m\'etricos compactos no necesariamente encajados en $M$.
Definimos a continuaci\'on, $\hat{\mathbb{X}}$ un estimador de $\mathbb{X}_{\mu}$
como una funci\'on de $X_{1},\dots,X_{n}$ que toma valores en el conjunto de espacios m\'etricos compactos.