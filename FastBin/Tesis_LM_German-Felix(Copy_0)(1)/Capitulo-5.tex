\chapter{Aspectos Estad\'isticos de la Homolog\'ia Persistente}\label{chap:Cap5}

Por s\'i sola, la homolog\'ia persistente no toma en cuenta la naturaleza estoc\'astica de los datos
y la variabilidad intrinseca de las cantidades topol\'ogicas que infieren.
Buscamos ahora un acercamiento esta\'distico a la homolog\'ia persistente,
considerando que los datos son generados de alguna distribuci\'on desconocida.
Comenzamos dando varos resultados de consistencia para la inferencia de la homolog\'ia persistente.

\section{Resultados de Consistencia para la Homolog\'ia Persistente}

Sup\'ongase que observamos $n$ puntos $\cpar{X_{1},\dots,X_{n}}$ en un espacio m\'etrico
$\cpar{M,\rho}$ obtenidas i. i. d. de una medida de probabilidad desconocida $\mu$ con
soporte compacto $\mathbb{X}_{\mu}$ la distancia de Gromov-Hausdorff nos permite comparar
$\mathbb{X}_{\mu}$ con otros espacios m\'etricos compactos no necesariamente encajados en $M$.
Definimos a continuaci\'on, $\hat{\mathbb{X}}$ un estimador de $\mathbb{X}_{\mu}$
como una funci\'on de $X_{1},\dots,X_{n}$ que toma valores en el conjunto de espacios m\'etricos compactos.

Sean $\mathrm{Filt}\cpar{\mathbb{X}_{\mu}}$ y $\mathrm{Filt}\cpar{\hat{\mathbb{X}}}$
dos filtraciones definidas en $\mathbb{X}_{\mu}$ y $\hat{\mathbb{X}}$.
Con el teorema \ref{teo:4.9} hemos visto que una estrategia natural
para estimar la homologi\'ia persistente de $\mathrm{Filt}\cpar{\mathbb{X}_{\mu}}$
consiste en estimar el soporte de $\hat{\mathbb{X}}$.
N\'otese que en algunos casos, el espacio $M$ puede ser desconocido
y las observaciones $X_{1},\dots,X_{n}$ solo se conocen mediante de sus distancias por pares
$\rho\cpar{X_{i}, X_{j}}$, $i$, $j = 1,\dots,n$.
La distancia de Gromov-Hausforff nos permite considerar el conjunto de observaciones
como un espacio m\'etrico abstracto de cardinalidad $n$,
independietemente de la manera en la que esta encajado en $M$.
Esta estructura general incluye el acercamiento m\'as est\'andar
que consiste en estamar el soporte con respecto a la distancia de Hausdorff
restringiendo los valores de $\hat{\mathbb{X}}$ a los conjuntos compactos en $M$.

El conjunto finito $\mathbb{X}_{n}:=\cllav{X_{1},\dots,X_{2}}$
es un estimador natural para el soporte de $\mathbb{X}_{\mu}$.
En muchos de los contextos que veremos a continuaci\'on,
$\mathbb{X}_{\mu}$ muestra tasas de convergencia \'optimas
con respecto a la ditancia de Hausdorff.
Para algunas constantes $a,b>0$, decimos que $\mu$ satisface el supuesto
$\cpar{a,b}$-est\'andar si para cualquier $x\in\mathbb{X}_{\mu}$ y cualquer $r>0$,
\begin{equation}
    \mu\cpar{B\cpar{x,r}}\geq\min\cpar{ar^{b},1}.
\end{equation}

Este supuesto es ampliamente usado en la literatura de la estimaci\'on de conjuntos
bajo la distancia de Hausdorff
(Cuevas y Rodriguez-Casal, 2004\cite{CuevasRodriguezCasal2004};
Singh et al., 2009\cite{Singh2009}).