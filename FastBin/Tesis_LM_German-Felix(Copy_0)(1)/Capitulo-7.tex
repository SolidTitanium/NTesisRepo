\chapter{Discusi\'on}

En este documento, porponemos un marco general de los m\'etodos m\'as est\'andar
en el \'area del an\'alisis topol\'ogico de datos.
Tambi\'en damos una presentaci\'on de los fundamentos matem\'aticos del ATD,
en aspectos topol\'ogicos, algebraicos, geom\'etricos y estad\'isticos.
La robustez de los m\'etodos del ATD (debido a su invariancia ante la deformaci\'on
e independencia de las coordenadas) y la representaci\'on comprimida de los datos que ofrece
hacen muy interesante su uso para el an\'alisis de datos, apredizaje autom\'atico y
la inteligencia artificial transparente.
Se han propuesto muchas aplicaciones en esta direcci\'on durante los ultimos a\~{n}os.
As\'i, el ATD forma parte de la caja de herramientas del cient\'ifico de datos.

Por supuesto, aunque el ATD este equipado para afrontar todo tipo de problemas,
aquellos que se aventuren en el uso de estos m\'etodos pueden encontrarse con
una variedad de problemas.
En los aspectos algoritmicos, calcular la homolog\'ia persistente puede ser costoso
en tiempo y recursos.
Aunque a\'un hay posibilidades de mejora, los recientes avances computacionales han
permitido que el ATD se convierta en un m\'etodo efectivo para el an\'alisis de datos,
gracias a librerias como GUDHI, por ejemplo.
Adem\'as, combinar el ATD usando m\'etodos de cuantizaci\'on, simplificaci\'on de graficas,
o reducci\'on de dimensionalidad, puede reducir el coste computacional de los algoritmos del
ATD de manera significativa.
Otro potencial problema que podemos encontrar es que volver a los datos para interpretar
las firmas topol\'ogicas puede llegar a ser complicado ya que estas firmas corresponden
a clases de equivalencia de ciclos.
Esto puede ser un problema al momento de identificar que parte de la nube de puntos
``A creado'' una cierta firma topol\'ogica.
Finalmente, la informaci\'on topol\'ogica y geom\'etrica que puede ser extraida de los datos
no es simepre eficaz para resolver cualquier problema dado en la ciencia de datos.
Combinar caracter\'isticas topol\'ogicas con otro tipo de descriptores es generalmente
el acercamiento adecuado.

Hoy en d\'ia, el ATD es un campo de investigaci\'on activo, relevante en muchos campos
cient\'ificos.
En particular, actualmente existe un inter\'es intenso por combinar de manera efectiva el
aprendizaje autom\'atico, la estad\'istica y el ATD.
En esta perspectiva, creemos que a\'un existe una necesidad de resultados estad\'isticos
que demuestren y cuantifiquen el inter\'es de estos acercamientos basados en el ATD.